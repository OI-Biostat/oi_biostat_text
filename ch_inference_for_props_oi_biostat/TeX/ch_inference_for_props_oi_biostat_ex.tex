%!TEX root=../../main.tex

\section{Exercises}
\label{infForPropExercises}

% Chapter 8 exercises
%
% many problems drawn from OI 4 source files
%__________________
\subsection{Inference for a single proportion}

% 1 ODD (OI4, 6.1)

\eoce{\qt{Vegetarian college students\label{veg_coll_students_CLT}} Suppose that 8\%
of college students are vegetarians. Determine if the following statements are
true or false, and explain your reasoning.
\begin{parts}
\item The distribution of the sample proportions of vegetarians in random
samples of size 60 is approximately normal since $n \ge 30$.
\item The distribution of the sample proportions of vegetarian college
students in random samples of size 50 is right skewed.
\item A random sample of 125 college students where 12\% are vegetarians
would be considered unusual.
\item A random sample of 250 college students where 12\% are vegetarians
would be considered unusual.
\item The standard error would be reduced by one-half if we increased the
sample size from 125 to~250.
\end{parts}

}{}

% 2 EVEN (OI4, 6.2)

\eoce{\qt{Young Americans, Part I\label{young_americans_CLT_1}} About 77\% of
young adults think they can achieve the American dream. Determine if the
following statements are true or false, and explain your reasoning.
\footfullcite{news:youngAmericans1}
\begin{parts}
\item The distribution of sample proportions of young Americans who think
they can achieve the American dream in samples of size 20 is left skewed.
\item The distribution of sample proportions of young Americans who think
they can achieve the American dream in random samples of size 40 is
approximately normal since $n \ge 30$.
\item A random sample of 60 young Americans where 85\% think they can achieve
the American dream would be considered unusual.
\item A random sample of 120 young Americans where 85\% think they can
achieve the American dream would be considered unusual.
\end{parts}
}{}

% 3 ODD (OI4, 6.5)

\eoce{\qt{Gender equality\label{gender_equality}}
The General Social Survey asked a random sample of
1,390 Americans the following question:
``On the whole, do you think it should or should not be
the government's responsibility to  promote equality
between men and women?''
82\% of the respondents said it ``should be''.
At a 95\% confidence level, this sample has 2\% margin of error.
Based on this information, determine if the following statements
are true or false, and explain your reasoning.\footfullcite{data:gss}
\begin{parts}
\item We are 95\% confident that between 80\% and 84\% of Americans in this
sample think it's the government's responsibility to promote equality between
men and women.
\item We are 95\% confident that between 80\% and 84\% of all Americans
think it's the government's responsibility to promote equality between
men and women.
\item If we considered many random samples of 1,390 Americans, and we calculated
95\% confidence intervals for each, 95\% of these intervals would include the
true population proportion of Americans who think it's the government's
responsibility to promote equality between men and women.
\item In order to decrease the margin of error to 1\%, we would need to
quadruple (multiply by 4) the sample size.
\item Based on this confidence interval, there is sufficient evidence to
conclude that a majority of Americans think it's the government's responsibility
to promote equality between men and women.
\end{parts}

% n = 1390
% should be: 1142
% p = 1142/1390 = 0.82
% me = sqrt(.82*.08/1390)*1.96 = 0.02
}{}

\textD{\newpage}

% 4 EVEN (OI4, 6.6)

\eoce{\qt{Elderly drivers\label{elderly_drivers_CI_concept}}
The Marist Poll published a report stating that 66\% of adults nationally
think licensed drivers should be required to retake their road test once
they reach 65 years of age. It was also reported that interviews were
conducted on 1,018 American adults, and that the margin of error was 3\%
using a 95\% confidence level. \footfullcite{data:elderlyDriving}
\begin{parts}
\item Verify the margin of error reported by The Marist Poll.
\item Based on a 95\% confidence interval, does the poll provide convincing
evidence that \textit{more than} 70\% of the population think that licensed
drivers should be required to retake their road test once they turn 65?
\end{parts}
}{}

% 5 ODD (OI4, 6.7)

\eoce{\qt{Fireworks on July 4$^{\text{th}}$\label{fireworks_CI_concept}} A local
news outlet reported that 56\% of 600 randomly sampled Kansas residents planned
to set off fireworks on July~$4^{th}$. Determine the margin of error for the
56\% point estimate using a 95\% confidence level.\footfullcite{data:july4}
}{}

% 6 EVEN (OI4, 6.8)

\eoce{\qt{Life rating in Greece\label{greece_life_rating_CI}} Greece has faced a
severe economic crisis since the end of 2009. A Gallup poll surveyed 1,000
randomly sampled Greeks in 2011 and found that 25\% of them said they would
rate their lives poorly enough to be considered ``suffering''.\footfullcite{data:suffering}
\begin{parts}
\item Describe the population parameter of interest. What is the value of the
point estimate of this parameter?
\item Check if the conditions required for constructing a confidence interval
based on these data are met.
\item Construct a 95\% confidence interval for the proportion of Greeks who
are ``suffering".
\item Without doing any calculations, describe what would happen to the
confidence interval if we decided to use a higher confidence level.
\item Without doing any calculations, describe what would happen to the
confidence interval if we used a larger sample.
\end{parts}
}{}

% 7 ODD (OI4, 6.9)

\eoce{\qt{Study abroad\label{study_abroad_CI_decision}}
A survey on 1,509 high school seniors who took the SAT
and who completed an optional web survey shows that
55\% of high school seniors are fairly certain that
they will participate in a study abroad program in
college.\footfullcite{data:studyAbroad}
\begin{parts}
\item
    Is this sample a representative sample from the population
    of all high school seniors in the US?
    Explain your reasoning.
\item
    Let's suppose the conditions for inference are met.
    Even if your answer to part (a) indicated that this approach
    would not be reliable, this analysis may still be interesting
    to carry out (though not report).
    Construct a 90\% confidence interval for the proportion of high
    school seniors (of those who took the SAT) who are fairly certain
    they will participate in a study abroad program in college,
    and interpret this interval in context.
\item
    What does ``90\% confidence" mean?
\item
    Based on this interval, would it be appropriate to claim that
    the majority of high school seniors are fairly certain that they
    will participate in a study abroad program in college?
\end{parts}
}{}

\textD{\newpage}

% 8 EVEN (OI4, 6.10)

\eoce{\qt{Legalization of marijuana, Part I\label{legalize_marijuana_CI_decision}}
The General Social Survey asked 1,578 US residents:
``Do you think the use of marijuana should be made legal, or not?''
61\% of the respondents said
it should be made legal.\footfullcite{data:gss}
\begin{parts}
\item Is 61\% a sample statistic or a population parameter? Explain.
\item Construct a 95\% confidence interval for the proportion of US
residents who think marijuana should be made legal, and interpret it in the
context of the data.
\item A critic points out that this 95\% confidence interval is only
accurate if the statistic follows a normal distribution, or if the normal
model is a good approximation. Is this true for these data? Explain.
\item A news piece on this survey's findings states, ``Majority of Americans
think marijuana should be legalized.'' Based on your confidence
interval, is this news piece's statement justified?
\end{parts}

% 2348 surveyed
% 770 not asked question
% 2348 - 770 = 1578 asked question
% 968 said legalize
% 968 / 1578 = 0.61

}{}

% 9 ODD (OI4, 6.11)

\eoce{\qt{National Health Plan, Part I\label{national_health_plan_HT}}
A \textit{Kaiser Family Foundation} poll for US adults
in 2019 found that 79\% of Democrats, 55\% of Independents,
and 24\% of Republicans supported a generic ``National Health Plan''.
There were 347 Democrats, 298 Republicans, and 617 Independents
surveyed.\footfullcite{data:KFF2019_nat_health_plan}
\begin{parts}
\item
    A political pundit on TV claims that a majority of Independents
    support a National Health Plan.
    Do these data provide strong evidence to support this type
    of statement?
\item
    Would you expect a confidence interval for the proportion
    of Independents who oppose the public option plan to
    include 0.5?
    Explain.
\end{parts}
}{}

% 10 EVEN (OI4, 6.16)

\eoce{\qt{Legalize Marijuana, Part II\label{legalize_marijuana_CI_sample_size}} As
	discussed in Exercise~\ref{legalize_marijuana_CI_decision},
	the General Social Survey reported a sample where about
	61\% of US residents thought marijuana should be made legal.
	If we wanted to limit the margin of error of
	a 95\% confidence interval to 2\%, about how many
	Americans would we need to survey?
}{}

% 11 ODD (OI4, 6.15)

\eoce{\qt{National Health Plan,
    Part II\label{national_health_plan_CI_sample_size_replaced}}
Exercise~\ref{national_health_plan_HT} presents the results
of a poll evaluating support for a generic
``National Health Plan'' in the US in 2019,
reporting that 55\% of Independents are supportive.
If we wanted to estimate this number to within 1\% with
90\% confidence, what would be an appropriate sample size?
}{}


% 12 EVEN (OI4, 6.44)

\eoce{\qt{Acetaminophen and liver damage\label{acetaminophen_CI_sample_size}} It
is believed that large doses of acetaminophen (the active ingredient in over
the counter pain relievers like Tylenol) may cause damage to the liver. A
researcher wants to conduct a study to estimate the proportion of
acetaminophen users who have liver damage. For participating in this study,
he will pay each subject \$20 and provide a free medical consultation if the
patient has liver damage.
\begin{parts}
\item If he wants to limit the margin of error of his 98\% confidence
interval to 2\%, what is the minimum amount of money he needs to set aside
to pay his subjects?
\item The amount you calculated in part (a) is substantially over his budget
so he decides to use fewer subjects. How will this affect the width of his
confidence interval?
\end{parts}
}{}


% 13 ODD (OI4, 6.43)

\eoce{\qt{College smokers\label{college_smokers_CI_sample_size}} We are interested
	in estimating the proportion of students at a university who smoke. Out of a
	random sample of 200 students from this university, 40 students smoke.
	\begin{parts}
		\item Calculate a 95\% confidence interval for the proportion of students at
		this university who smoke, and interpret this interval in context.
		(Reminder: Check conditions.)
		\item If we wanted the margin of error to be no larger than 2\% at a 95\%
		confidence level for the proportion of students who smoke, how big of a
		sample would we need?
	\end{parts}
}{}

\textD{\newpage}

% 14 EVEN (OI4, 6.48)

\eoce{\qt{2010 Healthcare Law\label{healthcare_CI_concept}} On June 28, 2012 the
U.S. Supreme Court upheld the much debated 2010 healthcare law, declaring it
constitutional. A Gallup poll released the day after this decision indicates
that 46\% of 1,012 Americans agree with this decision. At a 95\% confidence
level, this sample has a 3\% margin of error. Based on this information,
determine if the following statements are true or false, and explain your
reasoning.\footfullcite{data:healthcare2010}
\begin{parts}
\item We are 95\% confident that between  43\% and 49\% of Americans in this
sample support the decision of the U.S. Supreme Court on the 2010 healthcare
law.
\item We are 95\% confident that between 43\% and 49\% of Americans support
the decision of the U.S. Supreme Court on the 2010 healthcare law.
\item If we considered many random samples of 1,012 Americans, and we
calculated the sample proportions of those who support the decision of the
U.S. Supreme Court, 95\% of those sample proportions will be between 43\% and
49\%.
\item The margin of error at a 90\% confidence level would be higher than 3\%.
\end{parts}
}{}

% 15 oi_biostat

\eoce{\qt{Oral contraceptive use, Part I\label{oc_use_approx}} In a study of 100 randomly sampled 18 year-old women in an inner city neighborhood, 15 reported that they were taking birth control pills.

\begin{parts}

	\item Can the normal approximation to the binomial distribution be used to calculate a confidence interval for proportion of women using birth control pills in this neighborhood? Explain your answer.

	\item Compute an approximate 95\% confidence interval for the population proportion of women age 18 in this neighborhood taking birth control pills.

	\item Does the interval from part (b) support the claim that, for the young women in this neighborhood, the percentage who use birth control is not significantly different from the national average of 5\%?  Justify your answer.

\end{parts}

}{}

% 16 oi_biostat

\eoce{\qt{Oral contraceptive use, Part II\label{oc_use_binom}} Suppose that the study were repeated in a different inner city neighborhood and that out of 50 randomly sampled 18-year-old women, 6 reported that they were taking birth control pills. The researchers would like to assess the evidence that the proportion of 18-year-old women using birth control pills in this neighborhood is greater than the national average of 5\%.

	\begin{parts}

		\item Can the normal approximation to the binomial distribution be used to conduct a hypothesis test of the null hypothesis that the proportion of women using birth control pills in this neighborhood is equal to 0.05? Explain your answer.

		\item State the hypotheses for the analysis of interest and compute the $p$-value.

		\item Interpret the results from part (b) in the context of the data.

	\end{parts}
}{}


\textD{\newpage}


\subsection{Inference for the difference of two proportions}

% 17 ODD (OI4, 6.17)

\eoce{\qt{Social experiment, Part I\label{social_experiment_conditions}} A ``social
experiment" conducted by a TV program questioned what people do when they see
a very obviously bruised woman getting picked on by her boyfriend. On two
different occasions at the same restaurant, the same couple was depicted. In
one scenario the woman was dressed ``provocatively'' and in the other
scenario the woman was dressed ``conservatively''. The table below shows how
many restaurant diners were present under each scenario, and whether or not
they intervened.
\begin{center}
\begin{tabular}{ll cc c}
            &               & \multicolumn{2}{c}{\textit{Scenario}} \\
\cline{3-4}
                            &       & Provocative   & Conservative  & Total \\
\cline{2-5}
\multirow{2}{*}{\textit{Intervene}} &Yes & 5   & 15  & 20    \\
                            &No     & 15      & 10  & 25 \\
\cline{2-5}
                            &Total  & 20      & 25  & 45 \\
\end{tabular}
\end{center}
Explain why the sampling distribution of the difference between the
proportions of interventions under provocative and conservative scenarios
does not follow an approximately normal distribution.
}{}

% 18 EVEN (OI4, 6.18)

\eoce{\qt{Heart transplant success\label{heart_transplant_conditions}} The Stanford
University Heart Transplant Study was conducted to determine whether an
experimental heart transplant program increased lifespan. Each patient
entering the program was officially designated a heart transplant candidate,
meaning that he was gravely ill and might benefit from a new heart. Patients
were randomly assigned into treatment and control groups. Patients in the
treatment group received a transplant, and those in the control group did
not. The table below displays how many patients survived and died in each
group. \footfullcite{Turnbull+Brown+Hu:1974}\vspace{-2mm}
\begin{center}
\begin{tabular}{rcc}
\hline
            & control   & treatment \\
\hline
alive       & 4         & 24 \\
dead        & 30        & 45 \\
\hline
\end{tabular}
\end{center}
Suppose we are interested in estimating the difference in survival rate between
the control and treatment groups using a confidence interval.
Explain why we cannot construct such an interval using the normal
approximation. What might go wrong if we constructed the confidence interval
despite this problem?
}{}

% 19 ODD (OI4, 6.21)

\eoce{\qt{National Health Plan,
    Part III\label{national_health_plan_CI_replaced}}
Exercise~\ref{national_health_plan_HT}
presents the results of a poll evaluating support for
a generically branded ``National Health Plan''
in the United States.
79\% of 347 Democrats and 55\% of 617 Independents
support a National Health Plan.
\begin{parts}
\item
    Calculate a 95\% confidence interval for the
    difference between the proportion of Democrats
    and Independents who support a National
    Health Plan $(p_{D} - p_{I})$, and interpret
    it in this context.
    We have already checked conditions for you.
\item
    True or false:
    If we had picked a random Democrat and a random
    Independent at the time of this poll, it is more
    likely that the Democrat would support the National
    Health Plan than the Independent.
\end{parts}
}{}

% 20 EVEN (OI4, 6.22)

\eoce{\qt{Sleep deprivation, CA vs. OR, Part I\label{sleep_OR_CA_CI}} According to
	a report on sleep deprivation by the Centers for Disease Control and Prevention,
	the proportion of California residents who reported insufficient rest or sleep
	during each of the preceding 30 days is 8.0\%, while this proportion is 8.8\%
	for Oregon residents. These data are based on simple random samples of 11,545
	California and 4,691 Oregon residents. Calculate a 95\% confidence interval
	for the difference between the proportions of Californians and Oregonians who
	are sleep deprived and interpret it in context of the data.\footfullcite{data:sleepCAandOR}
}{}


\textD{\newpage}

% 21 oi_biostat, spring 2020 final exam

\eoce{\qt{Remdesivir in COVID-19\label{remdesivir_prop}} Remdesivir is an antiviral drug previously tested in animal models infected with coronaviruses like SARS and MERS. As of May 2020, remdesivir has temporary approval from the FDA for use in severely ill COVID-10 patients. A randomized controlled trial conducted in China enrolled 236 patients with severe COVID-19; 158 were assigned to receive remdesivir and 78 to receive a placebo. In the remdesivir group, 103 patients showed clinical improvement; in the placebo group, 45 patients showed clinical improvement.

	\begin{parts}

	\item Conduct a formal comparison of the clinical improvement rates and summarize your findings.

	\item Report and interpret an appropriate confidence interval.

	\end{parts}
}{}


% 22 EVEN (OI4, 6.24)

\eoce{\qt{Sleep deprivation, CA vs. OR, Part II\label{sleep_OR_CA_HT}}
Exercise~\ref{sleep_OR_CA_CI} provides data on sleep deprivation rates of
Californians and Oregonians. The proportion of California residents who
reported insufficient rest or sleep during each of the preceding 30 days is
8.0\%, while this proportion is 8.8\% for Oregon residents. These data are
based on simple random samples of 11,545 California and 4,691 Oregon
residents.
\begin{parts}
\item Conduct a hypothesis test to determine if these data provide strong
evidence the rate of sleep deprivation is different for the two states.
(Reminder: Check conditions)
\item It is possible the conclusion of the test in part (a) is incorrect. If
this is the case, what type of error was made?
\end{parts}
}{}


% 23 ODD (OI4, 6.19)

\eoce{\qt{Gender and color preference\label{gender_color_preference_CI_concept}}
	A study asked 1,924 male and 3,666 female undergraduate college students
	their favorite color.
	A 95\% confidence interval for the difference between
	the proportions of males and females whose favorite color is black
	$(p_{male} - p_{female})$ was calculated to be (0.02, 0.06).
	Based on this
	information, determine if the following statements are true or false, and
	explain your reasoning for each statement you identify as false.
	\footfullcite{Ellis:2001}
	\begin{parts}
		\item We are 95\% confident that the true proportion of males whose favorite
		color is black is 2\% lower to 6\% higher than the true proportion of females
		whose favorite color is black.
		\item We are 95\% confident that the true proportion of males whose favorite
		color is black is 2\% to 6\% higher than the true proportion of females whose
		favorite color is black.
		\item 95\% of random samples will produce 95\% confidence intervals that
		include the true difference between the population proportions of males and
		females whose favorite color is black.
		\item We can conclude that there is a significant difference between the
		proportions of males and females whose favorite color is black and that the
		difference between the two sample proportions is too large to plausibly be
		due to chance.
		\item The 95\% confidence interval for $(p_{female} - p_{male})$ cannot be
		calculated with only the information given in this exercise.
	\end{parts}
}{}

\textD{\newpage}

% 24 EVEN (OI4, 6.28)

\eoce{\qt{Prenatal vitamins and Autism\label{prenatal_vitamin_autism_HT}}
Researchers studying the link between prenatal vitamin use and autism
surveyed the mothers of a random sample of children aged 24 - 60 months with
autism and conducted another separate random sample for children with typical
development. The table below shows the number of mothers in each group who
did and did not use prenatal vitamins during the three months before
pregnancy (periconceptional period).\footfullcite{Schmidt:2011}\vspace{-1.8mm}
\begin{center}
\begin{tabular}{l l c c c}
		&			& \multicolumn{2}{c}{\textit{Autism}}	&		\\
\cline{3-4}
		&			& Autism		& Typical development		& Total	\\
\cline{2-5}
\textit{Periconceptional}	& No vitamin	& 111	& 70		& 181	\\
\textit{prenatal vitamin}	& Vitamin	& 143		& 159		& 302	\\
\cline{2-5}
							& Total		& 254		& 229		& 483
\end{tabular}
\end{center}\vspace{-4.2mm}
\begin{parts}
\item State appropriate hypotheses to test for independence of use of
prenatal vitamins during the three months before pregnancy and autism.
\item Complete the hypothesis test and state an appropriate conclusion.
(Reminder: Verify any necessary conditions for the test.)
\item A New York Times article reporting on this study was titled ``Prenatal
Vitamins May Ward Off Autism". Do you find the title of this article to be
appropriate? Explain your answer. Additionally, propose an alternative title.
\footfullcite{news:prenatalVitAutism}
\end{parts}
}{}

% 25 ODD (OI4, 6.27)

\eoce{\qt{Sleep deprived transportation workers\label{sleep_deprived_driver_HT}}
	The National Sleep Foundation conducted a survey on the sleep habits of
	randomly sampled transportation workers and a control sample of non-transportation
	workers. The results of the survey are shown below.
	\footfullcite{data:sleepTransport}\vspace{-1.8mm}
	\begin{center}
		\begin{tabular}{l c c c c c }
			& 			& \multicolumn{4}{c}{\textit{Transportation Professionals}} \\
			\cline{3-6}
			& 			& 		& Truck	& Train		& Bus/Taxi/Limo		\\
			& \textit{Control}& Pilots	& Drivers	& Operators	& Drivers	\\
			\cline{1-6}
			Less than 6 hours of sleep	& 35		& 19		& 35	& 29	& 21	\\
			6 to 8 hours of sleep		& 193		& 132	    & 117	& 119	& 131	\\
			More than 8 hours			& 64		& 51		& 51	& 32	& 58	\\
			\cline{1-6}
			Total						& 292		& 202	    & 203	& 180	& 210
		\end{tabular}
	\end{center}\vspace{-1.2mm}
	Conduct a hypothesis test to evaluate if these data provide evidence of a
	difference between the proportions of truck drivers and non-transportation
	workers (the control group) who get less than 6 hours of sleep per day, i.e.
	are considered sleep deprived.
}{}

% 26 EVEN (OI4, 6.30)

\eoce{\qt{An apple a day keeps the doctor
    away\label{apple_doctor_HT_concept}}
A physical education teacher at a high school wanting
to increase awareness on issues of nutrition and health
asked her students at the beginning of the semester
whether they believed the expression
``an apple a day keeps the doctor away'',
and 40\% of the students responded yes.
Throughout the semester she started each class with
a brief discussion of a study highlighting positive
effects of eating more fruits and vegetables.
She conducted the same apple-a-day survey at the end
of the semester, and this time 60\% of the students
responded yes.
Can she used a two-proportion method from this section
for this analysis?
Explain your reasoning.
}{}


\textD{\newpage}


\subsection{Inference for two or more groups}

% 27 ODD (OI4, 6.31)

\eoce{\qt{True or false, Part I\label{tf_chisq_1}} Determine if the statements below
	are true or false. For each false statement, suggest an alternative wording to
	make it a true statement.
	\begin{parts}
		\item The chi-square distribution, just like the normal distribution, has two
		parameters, mean and standard deviation.
		\item The chi-square distribution is always right skewed, regardless of the
		value of the degrees of freedom parameter.
		\item The chi-square statistic is always positive.
		\item As the degrees of freedom increases, the shape of the chi-square
		distribution becomes more skewed.
	\end{parts}
}{}

% 28 EVEN (OI4, 6.32)

\eoce{\qt{True or false, Part II\label{tf_chisq_2}} Determine if the statements below
	are true or false. For each false statement, suggest an alternative wording to
	make it a true statement.
	\begin{parts}
		\item As the degrees of freedom increases, the mean of the chi-square
		distribution increases.
		\item If you found $\chi^2 = 10$ with $df = 5$ you would fail to reject $H_0$
		at the 5\% significance level.
		\item When finding the p-value of a chi-square test, we always shade the tail
		areas in both tails.
		\item As the degrees of freedom increases, the variability of the chi-square
		distribution decreases.
	\end{parts}
}{}


% 29 ODD (OI4, 6.35)

\eoce{\qt{Quitters\label{quitters_chisq_independence}} Does being part of a
	support group affect the ability of people to quit smoking? A county
	health department enrolled 300 smokers in a randomized experiment. 150
	participants were assigned to a group that used a nicotine patch and
	met weekly with a support group; the other 150 received the patch and
	did not meet with a support group. At the end of the study, 40 of the
	participants in the patch plus support group had quit smoking while
	only 30 smokers had  quit in the other group.
	\begin{parts}
		\item Create a two-way table presenting the results of this study.
		\item Answer each of the following questions under the null hypothesis
		that being part of a support group does not affect the ability of
		people to quit smoking, and indicate whether the expected values are
		higher or lower than the observed values.
		\begin{subparts}
			\item How many subjects in the ``patch + support" group would you
			expect to quit?
			\item How many subjects in the ``patch only" group would you expect to
			not quit?
		\end{subparts}
	\end{parts}
}{}

% 30 EVEN (OI4, 6.38)

\eoce{\qt{Parasitic worm\label{parasitic_worm_chisq}}
	Lymphatic filariasis is a disease caused by a parasitic worm.
	Complications of the disease can lead to extreme swelling
	and other complications.
	Here we consider results from a randomized experiment
	that compared three
	different drug treatment options to clear people of the
	this parasite, which people are working to eliminate entirely.
	The results for the second year of the study are
	given below:\footfullcite{King_Suamani_2018}
	\begin{center}
		\begin{tabular}{l cc}
			\hline
			& Clear at Year 2 & Not Clear at Year 2 \\
			\hline
			Three drugs & 52 & 2 \\
			Two drugs & 31 & 24 \\
			Two drugs annually & 42 & 14 \\
			\hline
		\end{tabular}
	\end{center}
	\begin{parts}
		\item\label{parasitic_worm_chisq_hyp}
		Set up hypotheses for evaluating
		whether there is any difference in the
		performance of the treatments,
		and also check conditions.
		\item
		Statistical software was used to run
		a chi-square test, which output:
		\begin{align*}
		&X^2 = 23.7
		&&df = 2
		&&\text{p-value} = \text{7.2e-6}
		\end{align*}
		Use these results to evaluate the hypotheses
		from part~(\ref{parasitic_worm_chisq_hyp}),
		and provide a conclusion
		in the context of the problem.
	\end{parts}
}{}

\textD{\newpage}

% 31 oi_biostat

\eoce{\qt{PREVEND, Part IV\label{prevend_statin_edu}} In the PREVEND data, researchers measured various features of study participants, including data on statin use and highest level of education attained. A two-way table of education level and statin use is shown below.

	% latex table generated in R 3.6.2 by xtable 1.8-4 package
	% Mon Jun 08 12:00:02 2020
	\begin{center}
		\begin{tabular}{rrrrrr}
			\hline
			& Primary & LowerSec & UpperSec & Univ & Sum \\
			\hline
			NonUser & 31 & 111 & 107 & 136 & 385 \\
			User & 20 & 46 & 27 & 22 & 115 \\
			Sum & 51 & 157 & 134 & 158 & 500 \\
			\hline
		\end{tabular}
\end{center}

\begin{parts}
	\item Set up hypotheses for evaluating whether there is an association between statin use and educational level.

	\item Check assumptions required for an analysis of these data.

	\item Statistical software was used to conduct a $\chi^2$ test: the test statistic is 19.054, with $p$-value 0.0027. Summarize the conclusions in context of the data, and be sure to comment on the direction of association.
\end{parts}

}{}


% 32 EVEN (OI4, 6.46)

\eoce{\qt{Diabetes and unemployment\label{diabetes_unemp_effect_size}} A
Gallup poll surveyed Americans about their employment status and whether or
not they have diabetes. The survey results indicate that 1.5\% of the 47,774
employed (full or part time) and 2.5\% of the 5,855 unemployed 18-29 year
olds have diabetes.\footfullcite{data:employmentDiabetes}
\begin{parts}
	\item Create a two-way table presenting the results of this study.
	\item State appropriate hypotheses to test for difference in proportions of
	diabetes between employed and unemployed Americans.
	\item The sample difference is about 1\%. If we completed the hypothesis
	test, we would find that the p-value is very small (about 0), meaning the
	difference is statistically significant. Use this result to explain the
	difference between statistically significant and practically significant
	findings.
\end{parts}
}{}

% 33 oi_biostat, spring 2020 final exam

\eoce{\qt{TB Treatment\label{tb_diabetes}} Tuberculosis (TB) is an infectious disease caused by the \textit{Mycobacterium tuberculosis} bacteria. Active TB can be cured by adhering to a treatment regimen of several drugs for 6-9 months. A major barrier to eliminating TB worldwide is failure to adhere to treatment; this is known as defaulting from treatment. A study was conducted in Thailand to identify factors associated with default from treatment. The study results indicate that out of 54 diabetic participants, 0 defaulted from treatment; out of 1,180 non-diabetic participants, 54 defaulted from treatment. Participants were recruited at health centers upon diagnosis of TB.

	\begin{parts}

		\item Create a two-way table presenting the results of this study.
		\item State appropriate hypotheses to test for difference in proportions of treatment default between diabetics and non-diabetics.
		\item Check assumptions. You may use a less stringent version of the success-failure condition: the expected number of successes per group should be greater than or equal to 5 (rather than 10).
		\item Formally test whether the proportion of patients who default from treatment differs between diabetics and non-diabetics. Summarize your findings.
	\end{parts}

}{}

\textD{\newpage}

% 34 EVEN (OI4, 6.50) edited

\eoce{\qt{Coffee and Depression\label{coffee_depression_chisq_indep}}
Researchers conducted a study investigating the relationship between
caffeinated coffee consumption and risk of depression in women. They
collected data on 50,739 women free of depression symptoms at the start
of the study in the year 1996, and these women were followed through
2006. The researchers used questionnaires to collect data on
caffeinated coffee consumption, asked each individual about physician-
diagnosed depression, and also asked about the use of antidepressants.
The table below shows the distribution of incidences of depression by
amount of caffeinated coffee consumption.\footfullcite{Lucas:2011}
\begin{adjustwidth}{-4em}{-4em}
	{\small
		\begin{center}
			\begin{tabular}{l  l rrrrrr}
				&  \multicolumn{1}{c}{}		& \multicolumn{5}{c}{\textit{Caffeinated coffee consumption}} \\
				\cline{3-7}
				&		& $\le$ 1	& 2-6	& 1	& 2-3	& $\ge$ 4	&   \\
				&		& cup/week	& cups/week	& cup/day	& cups/day	& cups/day	& Total  \\
				\cline{2-8}
				\textit{Clinical} & Yes	& 670 & \fbox{\textcolor{oiB}{373}}	& 905	& 564	& 95 	& 2,607 \\
				\textit{depression}	& No& 11,545	& 6,244	& 16,329	& 11,726	& 2,288 	& 48,132 \\
				\cline{2-8}
				& Total	& 12,215	& 6,617 & 17,234	& 12,290	& 2,383 	& 50,739 \\
				\cline{2-8}
			\end{tabular}
		\end{center}
	}
\end{adjustwidth}
\begin{parts}
	\item What type of test is appropriate for evaluating if there is an
	association between coffee intake and depression?
	\item Write the hypotheses for the test you identified in part (a).
	\item Calculate the overall proportion of women who do and do not
	suffer from depression.
	\item Identify the expected count for the highlighted cell, and
	calculate the contribution of this cell to the test statistic.
	\item The test statistic is $\chi^2=20.93$. What is the p-value?
	\item What is the conclusion of the hypothesis test?
	\item One of the authors of this study was quoted on the NYTimes as
	saying it was ``too early to recommend that women load up on extra
	coffee" based on just this study.\footfullcite{news:coffeeDepression}
	Do you agree with this statement? Explain your reasoning.
\end{parts}
}{}

% 35 oi_biostat

\eoce{\qt{Mosquito nets and malaria\label{nets_malaria}} This problem examines a hypothetical prospective study about an important problem in the developing world: the use of mosquito nets to prevent malaria in children.  The nets are typically used to protect children from mosquitoes while sleeping.

	Suppose that in a large region of an African country, 100 households with one child are randomized to receive free mosquito nets for the child in the household and 100 households with one child are randomized to a control group where families do not receive the nets.

	You are given the following information:

	\begin{itemize}

		\item In the 100 households receiving the nets, 22 children became infected with malaria.

		\item In the 100 households without the nets, 30 children became infected with malaria.

		\item The 200 families selected to participate in the study may be regarded as a random sample from the families in the region, so the 100 families in each group may be regarded as random samples from the population.

		\item Malaria among children is common in this region, with a prevalence of approximately 25\%.

	\end{itemize}

	\begin{parts}

		\item  Write down the $2 \times 2$ contingency table that corresponds to the data from the trial, labeling the table clearly and including the row and column totals.

		\item  Under the hypothesis of no association between use of a mosquito net and malaria infection, calculate the expected number of infected children among 100 families who did receive a net.

		\item The $\chi^2$ statistic for this $2 \times 2$ table is 1.66.  Use this information to conduct a test of the null hypothesis of no effect of the use of a mosquito net on malaria infection in children.

		\item  Compute and interpret the estimated relative risk of malaria infection, comparing the households without a net to those with a net.

	\end{parts}

}{}

\textD{\newpage}

% 36 oi_biostat

\eoce{\qt{Health care fraud\label{health_care_fraud}} Most errors in billing insurance providers for health care services involve honest mistakes by patients, physicians, or others involved in the health care system.  However, fraud is a serious problem. The National Health Care Anti-Fraud Association estimates that approximately \$68 billion is lost to health care fraud each year. Often when fraud is suspected, an audit of randomly selected billings is conducted.  The selected claims are reviewed by experts and each claim is classified as allowed or not allowed.  The claims not allowed are considered to be potentially fraudulent.

	In general, the distribution of claims is highly skewed such that the majority of claims filed are small claims and only a few are large claims.  Since simple random sampling would likely be overwhelmed by small claims, claims chosen for auditing are sampled in a stratified way: a set number of claims are sampled from each category of claim size: small, medium, and large.  Here are data from an audit that used stratified sampling from three strata based on the claim size (i.e., monetary amount of the claim).

	\begin{center}
		\begin{tabular}{c|c|c}
			\hline
			\textbf{Stratum} & \textbf{Sampled Claims} & \textbf{Not Allowed} \\
			\hline
			Small & 100 & 10 \\
			Medium & 50 & 17 \\
			Large & 20 & 4 \\
			\hline
		\end{tabular}
	\end{center}

	\begin{parts}

		\item Can these data be used to estimate the proportion of large claims for which fraud might be expected?

		\item Can these data be used to estimate the proportion of possibly fraudulent claims that are large claims?

		\item Construct a $2 \times 3$ contingency table of counts for these data
		and include the marginal totals, with the rows being the classification of claims and the columns being the size of the claim.

		\item Calculate the expected number of claims that would not be allowed among the large claims, under the hypothesis of no association of between size of claim and the claim not being allowed.

		\item Is the use of the chi-square statistic justified for these data?

		\item A chi-square test of no association between size of claim and whether it was allowed has value 12.93. How many degrees of freedom does the chi-square statistic have and what is the $p$-value for a test of no association?

		\item Compute the $\chi^2$ residuals. Based on the residuals, interpret the findings in the context of the data.

	\end{parts}

}{}


% 37 oi_biostat, from unit 8 lab 2 (104, spring 2020)

\eoce{\qt{Anxiety\label{anxiety_fisher}} Psychologists conducted an experiment to investigate the effect of anxiety on a person's desire to be alone or in the company of others (Schacter 1959; Lehmann 1975). A group of 30 individuals were randomly assigned into two groups; one group was designated the "high anxiety" group and the other the "low anxiety" group. Those in the high-anxiety group were told that in the "upcoming experiment", they would be subjected to painful electric shocks, while those in the low-anxiety group were told that the shocks would be mild and painless\footnote{Individuals were not actually subjected to electric shocks of any kind}. All individuals were informed that there would be a 10 minute wait before the experiment began, and that they could choose whether to wait alone or with other participants.

 The following table summarizes the results:

\begin{center}
	\begin{tabular}{rrr|r}
		\hline
		& Wait Together & Wait Alone & Sum \\
		\hline
		High-Anxiety & 12 & 5 & 17 \\
		Low-Anxiety & 4 & 9 & 13 \\
		\hline
		Sum & 16 & 14 & 30 \\
		\hline
	\end{tabular}
\end{center}

\begin{parts}
	\item Under the null hypothesis of no association, what are the expected cell counts?

	\item Under the assumption that the marginal totals are fixed and the null hypothesis is true, what is the probability of the observed set of results?

	\item Enumerate the tables that are more extreme than what was observed, in the same direction.

	\item Conduct a formal test of association for the results and summarize your findings. Let $\alpha = 0.05$.
\end{parts}

}{}

\textD{\newpage}

% 38 oi_biostat pset 7, spring 2020 (104)

\eoce{\qt{Salt intake and CVD\label{salt_cvd_fisher}} Suppose we are interested in investigating the relationship between high salt intake and death from cardiovascular disease (CVD). One possible study design is to identify a group of high- and low-salt users then follow them over time to compare the relative frequency of CVD death in the two groups. In contrast, a less expensive study design is to look at death records, identify CVD deaths from non-CVD deaths, collect information about the dietary habits of the deceased, then compare salt intake between individuals who died of CVD versus those who died of other causes. This design is called a retrospective design.

	Suppose a retrospective study is done in a specific county of Massachusetts; data are collected on men ages 50-54 who died over a 1-month period. Of 35 men who died from CVD, 5 had a diet with high salt intake before they died, while of the 25 men who died from other causes, 2 had a diet with high salt intake. These data are summarized in the following table.

	\begin{center}
		\begin{tabular}{l|cc|c}
			& \textbf{CVD Death} & \textbf{Non-CVD Death} & \textbf{Total} \\ \hline
			\textbf{High Salt Diet} & 5 & 2 & 7  \\
			\textbf{Low Salt Diet} & 30 & 23 & 53 \\ \hline
			\textbf{Total} & 35 & 25 & 60  \\
		\end{tabular}\\
	\end{center}

	\begin{parts}

		\item Under the null hypothesis of no association, what are the expected cell counts?

		\item Of the 35 CVD deaths, 5 were in the high salt diet group and 30 were in the low salt diet group. Under the assumption that the marginal totals are fixed, enumerate all possible sets of results (i.e., the table counts) that are more extreme than what was observed, in the same direction.

		\item Calculate the probability of observing each set of results from part (b).

		\item Evaluate the statistical significance of the observed data with a two-sided alternative. Let $\alpha = 0.05$. Summarize your results.
	\end{parts}

}{}


\subsection{Chi-square tests for the fit of a distribution}

% 39 ODD (OI4, 6.33)

\eoce{\qt{Open source textbook\label{opensource_text_chisq_GOF}} A professor using
an open source introductory statistics book predicts that 60\% of the
students will purchase a hard copy of the book, 25\% will print it out from
the web, and 15\% will read it online. At the end of the semester he asks his
students to complete a survey where they indicate what format of the book
they used. Of the 126 students, 71 said they bought a hard copy of the book,
30 said they printed it out from the web, and 25 said they read it online.
\begin{parts}
	\item State the hypotheses for testing if the professor's predictions were
	inaccurate.
	\item How many students did the professor expect to buy the book, print the
	book, and read the book exclusively online?
	\item This is an appropriate setting for a chi-square test. List the
	conditions required for a test and verify they are satisfied.
	\item Calculate the chi-squared statistic, the degrees of freedom associated
	with it, and the p-value.
	\item Based on the p-value calculated in part (d), what is the conclusion of
	the hypothesis test? Interpret your conclusion in this context.
\end{parts}
}{}

% 40 EVEN (OI4, 6.34) edited

\eoce{\qt{Barking Deer\label{barking_deer_chisq_gof}} Microhabitat factors associated with foraging sites of barking deer in Hainan Island, China were examined. In this region, woods make up 4.8\% of the land, cultivated grass plots make up 14.7\%, and deciduous forests make up 39.6\%. Of the 426 sites where the deer forage, 4 were categorized as woods, 16 as cultivated grass plots, and 61 as deciduous forests. The table below summarizes these data.\footfullcite{teng:2004}
\begin{center}
	\begin{tabular}{c c c c c}
		woods	& cultivated grassplot	& deciduous forests	 & other & total \\
		\hline
		4		& 16					& 61			     & 345	 & 426 \\
	\end{tabular}
\end{center}

\begin{parts}
	\item Write the hypotheses for testing if barking deer prefer to forage in
	certain habitats over others.
	\item Check if the assumptions and conditions required for testing these hypotheses are reasonably met.
	\item Do these data provide convincing evidence that barking deer prefer to
	forage in certain habitats over others? Conduct an analysis and summarize your findings.
\end{parts}
}{}


\textD{\newpage}


\subsection{Outcome-based sampling: case-control studies}

% 41 oi_biostat

\eoce{\qt{CVD and Diabetes\label{cvd_diabetes_outcome}} An investigator asked for the records of patients diagnosed with diabetes in his practice, then sampled 20 patients with cardiovascular disease (CVD) and 80 patients without CVD. For the sampled patients, he then recorded whether or not the age of onset of diabetes was at age 50 or younger.  Of the 40 patients whose age of onset of diabetes was 50 years of age or earlier, 15 had cardiovascular disease.  In the remaining 60 patients, 5 had cardiovascular disease.

\begin{parts}

	\item Write the contingency table that summarizes the result of this study.

	\item What is the relative odds of cardiovascular disease, comparing the older patients to those less than 50 years old at onset of diabetes?

	\item Interpret the relative odds of cardiovascular disease and comment on whether the relative odds cohere with what you might expect.

	\item In statistical terms, state the null hypothesis of no association between the presence of cardiovascular disease and age of onset of diabetes.

	\item What test can be used to test the null hypothesis?  Are the assumptions for the test reasonably satisfied?

	\item The value of the chi-square test statistic for this table is 11. Identify the logical flaw in the following statement: "In this retrospective study of cardiovascular disease and diabetes, our study has demonstrated statistically significant evidence that diabetes increases the risk of cardiovascular disease."

\end{parts}

}{}

% 42 oi_biostat

\eoce{\qt{Blood thinners\label{blood_thinners}} Cardiopulmonary resuscitation (CPR) is a procedure commonly used on individuals suffering a heart attack when other emergency resources are not available. This procedure is helpful in maintaining some blood circulation, but the chest compressions involved can also cause internal injuries.  Internal bleeding and other injuries complicate additional treatment efforts following arrival at a hospital. For instance, while blood thinners may be used to help release a clot that is causing a heart attack, the blood thinner would have negative repercussions on any internal injuries.

	This problem uses data from a study in which patients who underwent CPR for a heart attack and were subsequently admitted to a hospital.  These patients were randomly divided into a treatment group where they received a blood thinner or the control group where they did not receive the blood thinner. The outcome variable of interest was whether the patients survived for at least 24 hours.

	The study results are shown in the table below:

		\begin{center}
			\begin{tabular}{l|cccc|c}
				\hline
				&& Treatment 	&  Control 	&& Total \\
				\hline
				Survived		&& 14		& 11		&& 25 \\
				Died		&&   26  	&  39	      && 65 \\
				\hline
				Total			&& 40		& 50		&& 90 \\
				\hline
			\end{tabular}
		\end{center}

		\begin{parts}
			\item For this table, calculate the odds ratio for survival, comparing treatment to control, and the relative risk of survival, comparing treatment to control.
			\item What is the interpretation of each of these two statistics?
			\item In this study, which of the two summary statistics in part (a) is the better description of the treatment effect?  Why?
		\end{parts}

}{}

\textD{\newpage}

% 43 oi_biostat

\eoce{\qt{CNS disorder\label{cns_disorder}} Suppose an investigator has studied the possible association between the use of a weight loss drug and a rare central nervous system (CNS) disorder.  He samples from a group of volunteers with and without the disorder, and records whether they have used the weight loss drug.  The data are summarized in the following table:

	\begin{center}
		\begin{tabular*}{0.30\textwidth}{r|c @{\extracolsep{\fill}} c @{\extracolsep{\fill}}}
			& \multicolumn{2}{c}{\textbf{Drug Use}}   \\ \hline
			\textbf {CNS disorder}& \textbf {Yes} & \textbf{No}  \\ \hline
			\textbf {Yes} & 10  & 2000   \\
			\textbf{No} &  7 & 4000  \\
		\end{tabular*}
	\end{center}

	\begin{parts}

		\item Can these data be used to estimate the probability of a CNS disorder for someone taking the weight loss drug?

		\item For this study, what is an appropriate measure of association between the weight-loss drug and the presence of CNS disorder?

		\item Calculate the measure of association specified in part (b).

		\item  Interpret the calculation from part (c).

		\item What test of significance is the best choice for analyzing the hypothesis of no association for these data?

	\end{parts}

}{}

% 44 oi_biostat spring 2020, final exam

\eoce{\qt{Asthma risk\label{asthma_animal}} Asthma is a chronic lung disease characterized as hypersensitivity to a variety of stimuli, such as tobacco smoke, mold, and pollen. The prevalence of asthma has been increasing in recent decades, especially in children. Some studies suggest that children who either live in a farm environment or have pets become less likely to develop asthma later in life, due to early exposure to elevated amounts of microorganisms. A large study was conducted in Norway to investigate the association between early exposure to animals and subsequent risk for asthma.

	Using data from national registers, researchers identified 11,585 children known to have asthma at age 6 years out of the 276,281 children born in Norway between January 1, 2006 and December 31, 2009. Children whose parents were registered as "animal producers and related workers" during the child's first year of life were defined as being exposed to farm animals. Of the 958 children exposed to farm animals, 19 had an asthma diagnosis at age 6.

	\begin{parts}

		\item Do these data support previous findings that living in a farm environment is associated with lower risk of childhood asthma? Conduct a formal analysis and summarize your findings. Be sure to check any necessary assumptions.

		\item Is the relative risk an appropriate measure of association for these data? Briefly explain your answer.

		\item In language accessible to someone who has not taken a statistics course, explain whether these results represent evidence that exposure to farm animals reduces the risk of developing asthma. Limit your answer to no more than seven sentences.
	\end{parts}

}{}

% 45 oi_biostat section 8 spring 2020

\eoce{\qt{Tea consumption and carcinoma\label{tea_carcinoma}} In a study examining the association between green tea consumption and esophageal carcinoma, researchers recruited 300 patients with carcinoma and 571 without carcinoma and administered a questionnaire about tea drinking habits. Out of the 47 individuals who reported that they regularly drink green tea, 17 had carcinoma. Out of the 824 individuals who reported they never drink green tea, 283 had carcinoma.

	\begin{parts}

		\item Analyze the data to assess evidence for an association between green tea consumption and esophageal carcinoma from these data. Summarize your results.

		\item Report and interpret an appropriate measure of association.

	\end{parts}

}{}

\subsection{Inference for two samples of binary data}

% these exercises are not currently in the q-bank

% 46

\eoce{\qt{Prevalence difference versus prevalence ratio, I}

  \begin{parts}

  \item Assume that the prevalence for a particular disease in two groups
    is 40.4\% and 42\%.  Calculate the prevalence difference and ratio for the disease, comparing the group with the higher prevalence to the one with the lower prevalence. For each summary measure, provide an interpretation that a non-statistician would understand.

  \item Now assume that the prevalence for a particular disease in two groups is 1.2\% and 2.8\%.  Calculate the prevalence difference and ratio for the disease, comparing the group with the higher prevalence to the one with the lower prevalence. For each summary measure, provide an interpretation that a non-statistician would understand.

  \end{parts}

}{}

% 47

\eoce{\qt{Prevalence difference versus prevalence ratio, II}

  \begin{parts}

  \item Assume that the prevalence for a particular disease in two groups
    is 10\% and 15\%.  Calculate the prevalence difference and ratio for the disease, comparing the group with the higher prevalence to the one with the lower prevalence. For each summary measure, provide an interpretation that a non-statistician would understand.

  \item Now assume that the prevalence for a particular disease in two groups is 40\% and 45\%.  Calculate the prevalence difference and ratio for the disease, comparing the group with the higher prevalence to the one with the lower prevalence. For each summary measure, provide an interpretation that a non-statistician would understand.

  \end{parts}
}{}

% 48

\eoce{\qt{Birth defects and paternal alcohol consumption} A 2021 study by Zhou et. al\footfullcite{zhou:2021}in JAMA Pediatrics discussed the possible association of congenital heart defects in a newborn and paternal alcohol consumption.  The study was described as a prospective study in  which the study team recruited more than 529,090 couples who were planning to become pregnant in the next 6 months, then recorded alcohol consumption and birth defects.  Of the participating couples, 364,939 fathers did not drink alcohol before conception (defined as at least one time drinking a week) and 164,151 did.  Among the fathers who consumed alcohol, there were 363 birth defects. Among the fathers who did not consume alcohol, there were 246 birth defects.

  \begin{parts}

  \item Should the analysis of this study use risk  or odds ratio as summary statistic for the association of paternal alcohol consumption and fetal birth defects.  Why?

  \item Calculate the statistic you have recommended in part (a).

  \item Calculate a 95\% confidence interval for the measure of association in part (a).

  \end{parts}
}{}

% 49

\eoce{\qt{High salt diet and cardiovascular disease related death} Suppose a retrospective study is done in a specific county of Massachusetts; data are collected on men ages 50-54 who died over a 1-month period. Of 35 men who died from CVD, 5 had a diet with high salt intake before they died, while of the 25 men who died from other causes, 2 had a diet with high salt intake. These data are summarized in the following table.
\vspace{0.2cm}
\begin{tabular}{l|cc|c}
   & \textbf{CVD Death} & \textbf{Non-CVD Death} & \textbf{Total} \\ \hline
  \textbf{High Salt Diet} & 5 & 2 & 7  \\
  \textbf{Low Salt Diet} & 30 & 23 & 53 \\ \hline
  \textbf{Total} & 35 & 25 & 60  \\
\end{tabular}
\\

\begin{parts}
\item In this study sample, what are the estimated odds that a male had a high salt diet?  A low salt diet?
\item Among the men where the recorded death was due to CVD, what are the odds that the male had a high salt diet?  What are the odds of a low salt diet in the same group?
\item What is the OR for a CVD related death, comparing a high  to a low salt diet?
\item What is the OR for a death not related to CVD, comparing a high to a low salt diet?
\end{parts}
}{}

% 50

% copied from logistic regression exercises
\eoce{\qt{Diabetes} In the United States, approximately 9\% of the population have diabetes.
  \begin{parts}
  \item What are the odds that a randomly selected member of the US population has diabetes?
  \item Suppose that in a primary care clinic, the prevalence of diabetes among the patients seen in the clinic is 12\%.  What is the probability that a randomly selected patient in the clinic has diabetes?  What are the odds of diabetes for that patient?
  \item If in a particular population the probability of diabetes is twice what it is in the general population, does the odds of diabetes double?
  \end{parts}
}{}

% 51

\eoce{\qt{Treatment for COVID-19, I\label{tofacitinibPlaceboRR}}
  Guimar\~{a}es, et al.\footfullcite{tofacitinib_covid19}  reported the results of a randomized trial comparing tofacitinib to placebo in patients in Brazil hospitalized with Covid-19 pneumonia.  Since there were no known effective treatments for Covid-19 pneumonia when the trial was conducted, a placebo control group was considered ethical.  Of the 145 participants assigned to placebo, 42 experienced the outcome of interest, death or respiratory failure during the 28 day follow-up period; 26 out of the 144 assigned to tofacitinib experienced the outcome.

 \begin{parts}

\item Calculate a 95\% confidence interval for the between group difference in the risk of death or respiratory failure.

\item Conduct a test of the hypothesis of no difference between the groups.

\item Calculate a 95\% confidence interval for the risk ratio, comparing tofacitinib to placebo.

\end{parts}

}{}

% 52

\eoce{\qt{Treatment for COVID-19, II\label{tofacitinibPlaceboChiSq}} Using the data in Problem~\ref{tofacitinibPlaceboRR}:

\begin{parts}

 \item Construct a $2 \times 2$ table summarizing the data, with the treatment variable in the rows and outcome in the columns.

 \item Calculate the expected cell counts under the null hypothesis of no treatment effect.  Are the conditions for the $\chi^2$-test met?

 \item Verify that the $\chi^2$-statistic has value 4.77.

\end{parts}
}{}

% 53

\eoce{\qt{Fisher's exact test, I}
Suppose the partial data in the following table summarize the results of a small randomized trial with 11 participants, in which 6 are assigned to control and 5 to treatment. Of those in the treatment group, 3 respond to treatment.
      \vspace{0.2cm}

      \centering
      \begin{tabular}{l rr |r}
        \hline
         & Response & No Response & Total \\
        \hline
        Treatment &  4 &  & 5    \\
        Control &  &  &  6   \\
        \hline
        Total & 5& 6 & 11 \\
        \hline
      \end{tabular}
      \flushleft
\begin{parts}

\item Show that the 4 in the upper left cell determines the counts in the rest of the table.

\item What is the relative risk for a response, comparing treatment  to control?

\item What are the tables that are as or more extreme whose results favor treatment?

\item Calculate the Fisher's exact test one-sided $p$-value for a test of the null hypothesis of no treatment effect on response.

\end{parts}
}{}

% 54

\eoce{\qt{Fisher's exact test, II}
Suppose the partial data in the following table summarize the results of a small randomized trial with 11 participants, in which 6 are assigned to control and 5 to treatment. Of those in the treatment group, 3 respond to treatment.
      \vspace{0.2cm}

      \centering
      \begin{tabular}{l rr |r}
        \hline
         & Response & No Response & Total \\
        \hline
        Treatment &  3 &  & 5    \\
        Control &  &  &  6   \\
        \hline
        Total & 5& 6 & 11 \\
        \hline
      \end{tabular}
      \flushleft
\begin{parts}

\item Show that the 3 in the upper left cell determines the counts in the rest of the table.

\item What is the relative risk for a response, comparing treatment  to control?

\item What are the tables that are as or more extreme whose results favor treatment?

\item Calculate the Fisher's exact test one-sided $p$-value for a test of the null hypothesis of no treatment effect on response.

\end{parts}
}{}
