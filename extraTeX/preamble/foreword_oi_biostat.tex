\chapter*{Foreword}

The past year has been challenging for the health sciences in ways that we could not have imagined when we started writing 5 years ago. The rapid spread of the SARS coronavirus (SARS-CoV-2) worldwide has upended the scientific research process and highlighted the need for maintaining a balance between speed and reliability. Major medical journals have dramatically increased the pace of publication; the urgency of the situation necessitates that data and research findings be made available as quickly as possible to inform public policy and clinical practice. Yet it remains essential that studies undergo rigorous review; the retraction of two high-profile coronavirus studies \footfullcite{mehra2020acearb}\footfullcite{mehra2020hydroxychloroquine}
sparked widespread concerns about data integrity, reproducibility, and the editorial process.

In parallel, deepening public awareness of structural racism has caused a re-examination of the role of race in published studies in health and medicine.  A recent review of algorithms used to direct treatment in areas such as cardiology, obstetrics and oncology uncovered examples of race used in ways that may lead to substandard care for people of color.\footfullcite{vyasHiddenPlainSight}  The SARS-CoV-2 pandemic has reminded us once again that marginalized populations are disproportionately at risk for bad health outcomes. Data on 17 million patients in England \footfullcite{williamson2020} suggest that Blacks and South Asians have a death rate that is approximately 50\% higher than white members of the population. 

Understanding the SARS coronavirus and tackling racial disparities in health outcomes are but two of the many areas in which Biostatistics will play an important role in the coming decades. Much of that work will be done by those now beginning their study of Biostatistics. We hope this book provides an accessible point of entry for students planning to begin work in biology, medicine, or public health. While the material presented in this book is essential for understanding the foundations of the discipline, we advise readers to remember that a mastery of technical details is secondary to choosing important scientific questions, examining data without bias, and reporting results that transparently display the strengths and weaknesses of a study.

